\documentclass{article}
\usepackage[utf8]{inputenc}
\usepackage{graphicx}
\usepackage{fancyhdr}
\pagestyle{fancy}
\lfoot{\texttt{comsm0094.github.io}}
\lhead{LC\&B - 03.2 I\&F - extra - Conor}
\rhead{\thepage}
\cfoot{}
\begin{document}

\section*{Note on the f-I curve}

\subsection*{Introduction}

The notes contain a figure, repeated here as Fig.~\ref{f_i_curve}
showing the f-I curve for an integrate and fire neuron. To save time
the shape of the curve was never derived but for the curious I have
writen it out here.

In the model 
\begin{equation}
\tau_m\frac{dV}{dt}=E_L-V+R_mI_e
\end{equation}
which we can solve from our study of odes, it gives
\begin{equation}
V(t)=E_L+R_mI_e+[V(0)-E_L-R_mI_e]e^{-t/\tau_m}
\end{equation}
so if the neuron has spiked and is reset at time $t=0$ and reaches
threshold at time $t=T$, assume $V_R=E_L$ we have
\begin{equation}
V_T=E_L+R_mI_e-R_mI_ee^{-T/\tau_m}
\end{equation}
so 
\begin{equation}
e^{-T/\tau_m}=\frac{E_L+R_mI_e-V_T}{R_mI_e}
\end{equation}
Taking the log of both sides we get
\begin{equation}
T=\tau_m\log\left[\frac{R_mI_e}{E_L+R_mI_e-V_T}\right]
\end{equation}
Finally, this is the time between spikes, so the frequency is one over this. It is only defined for $R_mI_e>V_T-E_L$, below that there is no spiking and the frequency is zero. The actual gnuplot command used to make the figure was\\
\texttt{plot [0:22] x>15 ? 1/(.01*log(x/(x-15))) : 0}

\begin{figure}
\begin{center}
\include{f_i_curve}
\end{center}
\caption{The firing rate, that is spikes per second, for the integrate
  and fire neuron with different constant inputs with $\tau_m=10$ ms,
  $V_T=-55$ mV and both the leak and reset given by $-70$ mV. Notice
  how there is no firing until a threshold is reached and after that
  the firing increases very quickly. \label{f_i_curve}}
\end{figure}

\end{document}

