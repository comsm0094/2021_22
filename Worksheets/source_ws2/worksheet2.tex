
\documentclass[12pt]{article}
\usepackage{amsfonts, epsfig}

\usepackage{amsmath}
\usepackage{graphicx}
\usepackage{fancyhdr}
\pagestyle{fancy}
\lfoot{\texttt{comsm0094.github.com}}
\lhead{Worksheet 2: FitzHugh-Nagumo model}
\rhead{\thepage}
\cfoot{}

\newcommand{\ms}{\mbox{\,ms}}
\newcommand{\mV}{\mbox{\,mV}}
\newcommand{\kHz}{\mbox{\,kHz}}

\begin{document}

\section*{The FitzHugh-Nagumo model}
To recap, we have seen the Hodgkin Huxley equation
\begin{equation}
C_m\frac{dV}{dt}=g_\ell(E_{\ell}-V)+g_{\text{Na}}(E_{\text{Na}}-V)+g_{\text{K}}(E_{\text{K}}-V)+I
\end{equation}
where $I$ is an external current. The sodium and potassium conductances, $g_{\text{Na}}$ and
$g_{\text{K}}$ have complicated non-linear dynamics. In fact, the
Hodgkin-Huxley equation is really four equation, an equation for $V$
along with equations for the three gating variables $n$, $m$ and $h$, each of the form
\begin{equation}
\frac{dk}{dt}=\alpha_k(1-k)-\beta_k k
\end{equation}
with $k$ standing in for $n$, $m$ or $h$ and $\alpha_k(V)$ and $\beta_k(V)$
being the probability of going from closed to open and open to
closed. By moving stuff around this can be easily rewritten in a familiar form
\begin{equation}
\tau_k\frac{dk}{dt}=k_\infty - k
\end{equation}
with 
\begin{equation}
\tau_k=\frac{1}{\alpha_k+\beta_k}
\end{equation}
and
\begin{equation}
k_\infty=\frac{\alpha_k}{\alpha_k+\beta_k}
\end{equation}
Hence the ion channels relax towards some asymptotic value $n_\infty$,
$m_\infty$ and $h_\infty$ with some time scale $\tau_n$, $\tau_m$ and
$\tau_h$; however all these quantities depend on the voltage so the
equations are coupled to the voltage equation.

This is a complicated non-linear equation; in fact, real neurons are
likely to be even more complicated. As was mentioned before; the
Hodgkin-Huxley equation as described here describes the voltage
dynamics of the squid giant axon. The same name is often used to
describe the more complex versions of these models
\begin{equation}
C_m\frac{dV}{dt}=\mbox{lots of different ionic currents}
\end{equation}
used to describe real neurons. These typically include other channels,
often more than one potassium channel which adds other timescales to
the dynamics, calcium channels which play some of the same role as the
sodium channels, other sodium channels with more complex dynamics,
calcium-gated potassium channels whose open and closing depends not on
the voltage but on the concentration of calium ions, and so on. All of
this complexity emphasises the neuron itself as a site of computation,
rather than as a node in a network. As such, it is important to
understand the dynamics of the neurons models, however, their
complexity makes this hard and so we start not by allowing the
Hodgkin-Huxley model to become more complex, but by simplifying it.

\subsection*{FitzHugh-Nagumo model}

The FitzHugh-Nagumo model \cite{FitzHugh1955,Nagumo1962} is a set of
two differential equations that show similar behaviour to the
Hodgkin-Huxley equations: they are held to capture some of the
dynamical structure of the Hodgkin-Huxley equations, something that
can be explored using phase analysis. Here, however, we are going to
restrict ourselves to solving the equations numerically and observing
the spiking.

The FitzHugh-Nagumo model is governed by the equations
\begin{eqnarray}
\frac{dv}{dt}&=&v-\frac{1}{3}v^3-w+I\cr
\tau\frac{dw}{dt}&=&v+a-bw
\end{eqnarray}
where a little-v, $v$, has been used for the voltage to show these
quantities shouldn't be take seriously as biologically relevant, they
have been scaled to, for example, get rid of one of the time
constants. The little-w, $w$, should be thought of as modelling the
net effect of the other currents in the Hodgkin-Huxley equations.

One application of this is the study of bursting cells and pattern
generation; these are cells that send out regular burst of neurons;
these dynamics are important in controlling some fundamental
physiological systems where patterns are important, chewing in slugs,
struggling in tadpoles and so on.

In this coursework we will solve the equation numerically with the
parameters $I=0.5$, $a=0.7$, $b=0.8$, and $\tau=12.5$. This is a bit
different from the previous example of the Euler method because now we
have two variables to update, \texttt{v} and \texttt{w}, corresponding
to $v$ and $w$. This isn't too complicated, each gets updated using
the derivative in the usual way.

One complication is that they get
updated `at the same time', in other words, you should do something like:
\begin{eqnarray*}
  \mathtt{v\_old}&=&\mathtt{v}\\
  \mathtt{w\_old}&=&\mathtt{w}
\end{eqnarray*}
and then
\begin{eqnarray*}
  \mathtt{v}&=&\mathtt{v\_derivative(v\_old,w\_old)*dt}\\
  \mathtt{w}&=&\mathtt{w\_derivative(v\_old,w\_old)*dt}
\end{eqnarray*}
so that both variables get updated using the values from the previous time step. If you just did
\begin{eqnarray*}
  \mathtt{v}&=&\mathtt{v\_derivative(v,w)*dt}\\
  \mathtt{w}&=&\mathtt{w\_derivative(v,w)*dt}
\end{eqnarray*}
then \texttt{v} would be updated using the old values, whereas \texttt{w} would be updated using the new value of \texttt{v}!

\begin{thebibliography}{99}

\bibitem{FitzHugh1955}
\newblock FitzHugh R. (1955) 
\newblock Mathematical models of threshold phenomena in the nerve membrane. 
\newblock Bull. Math. Biophysics, 17:257--278

\bibitem{Nagumo1962}
\newblock Nagumo J., Arimoto S., and Yoshizawa S. (1962) 
\newblock An active pulse transmission line simulating nerve axon. 
\newblock Proc. IRE. 50:2061--2070.

\end{thebibliography}

\end{document}



