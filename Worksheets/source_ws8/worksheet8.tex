%Conor Houghton conor.houghton@bristol.ac.uk

%To the extent possible under law, the author has dedicated all copyright 
%and related and neighboring rights to these notes to the public domain 
%worldwide. These notes are distributed without any warranty. 


\documentclass[11pt,a4paper]{scrartcl}
\typearea{12}
\usepackage{graphicx}
%\usepackage{pstricks}
\usepackage{listings}
\usepackage{color}
\usepackage{tikz}
\usetikzlibrary{decorations.markings}
\lstset{language=C}
\usepackage{fancyhdr}
\pagestyle{fancy}
\lhead{\texttt{comsm0094.github.io}}
\lfoot{COMSM0094 - LCB WS8 - Conor}
\begin{document}

\section*{Worksheet 8}

\subsection*{Useful facts}

\begin{itemize}

\item The \textbf{conditional probability} of event $R$ given $C$:
  \begin{equation}
P(R|C)=\frac{P(R\cap C)}{P(C)} 
  \end{equation}
  This is the probability of getting an outcome in event $R$ if we
  know the outcome is in event $C$.

\item \textbf{Independence}: two events $A$ and $B$ are \textbf{independent} iff
  \begin{equation}
    P(A\cap B)=P(A)P(B)
  \end{equation}
  They are \textbf{conditionally independent}, given $C$, iff
  \begin{equation}
    P(A\cap B|C)=P(A|C)P(B|C)
  \end{equation}

\item \textbf{Bayes's rule}
\begin{equation}
P(A|B)=\frac{P(B|A)P(A)}{P(B)}
\end{equation}

\item \textbf{Set notation}:
  \begin{itemize}
\item The bar `$|$' in sets should be read as `such that', so $A=\{x|\mbox{some stuff}\}$ should be read as $A$ is the set of $x$ \textbf{such that} `some stuff' is true and $A=\{x\in \mathbf{Z}|x>3\mbox{ and }x<10\}$ is the set $A=\{4,5,6,7,8,9\}$. $\mathbf{Z}$ by the way is the set of integers.
  \item $A\cup B$ is the union so $A\cup B=\{x|x\in A\mbox{ or }x\in B\}$. If $A=\{1,2,3\}$ and $B=\{3,4,5\}$ then $A\cup B=\{1,2,3,4,5\}$
  \item $A\cap B$ is the intersection so $A\cap B=\{x|x\in A\mbox{ and }x\in B\}$. If $A=\{1,2,3\}$ and $B=\{3,4,5\}$ then $A\cup B=\{3\}$
   \item $A\setminus B$ is the set minus so $A\setminus B=\{x|x\in A\mbox{ and }x\not\in B\}$. If $A=\{1,2,3,4\}$ and $B=\{1,3,5\}$ then $A\setminus B=\{2,4\}$
   \item If $C$ is a subset, the complement of $C$, that is the set of all the elements not in $C$, is written $\bar{C}$. If $X=\{1,2,3,4\}$ and $C=\{1,2\}$ then $\bar{C}=\{3,4\}$.
  \end{itemize}
  For events, $A\cup B$ is the event of $A$ or $B$ happening, $A\cap
  B$ is the event of $A$ and $B$ happening, $A\setminus B$ is the
  event of $A$ happening but $B$ not happening and $\bar{C}$ is the
  event of $C$ not happening.

  \end{itemize}

\subsection*{Questions}

This is an opportunity to revise the basics of using Bayes's rule!

\begin{enumerate}

\item In a library where all books have blue or yellow spines, four
  fifths of books with yellow spines are about mathematics but only a
  fifth of books with blue spines are about mathematics. There are the
  same number of yellow and blue spined books, you come upom a book
  open on a table; the book is about mathematics. What is the chance
  it has a yellow spine?

\item You want to go for a walk. However, when you wake up the day is
  cloudy and half of all raining days start off cloudy. On the other
  hand, two days in five start off cloudy and it's been rather dry
  recently with only rain only on one day in ten. What is the chance
  it will rain?
  
\item One night in a bar in Las Vegas you meet a dodgy character who
  tells you that there are two types of slot machine in the Topicana,
  one that pays out 10\% of the time, the other 20\%. One sort of
  machine is blue, the other red. Unfortunately the dodgy character is
  too drunk to remember which is which. The next day you randomly
  select red to try, you find a red machine and put in a coin. You
  lose. Assuming the dodgy character was telling the truth, what is
  the chance the red machine is the one that pays out more. If you had
  won instead of losing, what would the chance be?\footnote{I stole
    this problem from \texttt{courses.smp.uq.edu.au/MATH3104/}}

\end{enumerate}

\end{document}

