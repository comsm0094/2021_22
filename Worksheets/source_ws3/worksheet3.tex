

\documentclass[12pt]{article}
\usepackage{amsfonts, epsfig}

\usepackage{amsmath}
\usepackage{graphicx}
\usepackage{fancyhdr}
\pagestyle{fancy}
\lfoot{\texttt{comsm0094.github.com}}
\lhead{Worksheet 3: Synapses}
\rhead{\thepage}
\cfoot{}

\newcommand{\ms}{\mbox{\,ms}}
\newcommand{\mV}{\mbox{\,mV}}
\newcommand{\kHz}{\mbox{\,kHz}}

\begin{document}

\section*{Synapses}

Simulate two neurons which have synaptic connections between each
other, that is the first neuron projects to the second, and the second
neuron projects to the first. Both model neurons should have the same
parameters: $\tau_m = 20$ ms, $E_L = -70$ mV, $V_r = -80$ mV, $V_t =
-54$ mV, $R_mI_e = 18$ mV and their synapses should also have the same
parameters: $R_m \bar{g}_s = 0.075$, $\tau_s= 10$ ms. For simplicity
take the synaptic conductance to satisfy
\begin{equation}
\tau_s\frac{ds}{dt}=-s
\end{equation}
with a spike arriving causing $s$ to increase by one, take the synapse
strength to be $g_s=\bar{g}_s s$. Simulate two cases: a) assuming that
the synapses are excitatory with $E_s = 0$ mV, and b) assuming that
the synapses are inhibitory with $E_s = -80$ mV. For each simulation
set the initial membrane potentials of the neurons $V$ to different
values chosen randomly from between $V_r$ and $V_t$ and simulate 1 s
of activity. For each case plot the voltages of the two neurons on the
same graph with different colours.

You'll notice that there is no seperate figure given for $R_m$, just
$R_m \bar{g}_s$ and $R_m I_e$, but, since the voltages for each neuron
satisfies an equation of the form
\begin{equation}
  \tau_m \frac{dV}{dt}= E_L-V+R_m\bar{g}_s s(E_s-V)+R_mI_e
\end{equation}
this is all that is needed.


\end{document}

